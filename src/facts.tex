\begin{enumerate}
	\setlength\itemsep{0em}
	\item  Раньше для работы с базами данных программисты использовали JDBC. 
	\item  ORM и JPA — новые инструменты для работы с БД, в которой были учтены пожелания программистов и который стал значительно проще в использовании нежели JDBC
	\item Для практики программист всю конфигурацию выносит в отдельный app.properties и добавляет часть кода для запуска SpringBoot-приложения, а также код, связанный с настройкой работы с БД, выносит в отдельный Java-config 
	\item В 2k16 наше время не нужно больше настраивать много нудных настроек для работы с БД в JDBC, достаточно создать DAO для сущности и можно приступать к работе. 
	\item В CrudRepository хранится набор методов для работы с сущностью: выборка (по ID/всех), удаление, сохранение и тд.
	\item В тестах мы можем использовать метод findUserByName(“test”), присвоив его к объекту User, который однозначно дает понять каким образом Spring формирует прокси-объекты 
	\item Многие возможности методов SpringData с их проблемами можно заменить на аннотации интерфейса Repository 
\end{enumerate}